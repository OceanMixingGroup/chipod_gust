\chapter{\texttt{combine\_turbulence}}

This is the final step.

\section{Masking criteria/thresholds}
\label{sec:orgccff08d}
\subsection{Background stratification: \texttt{min\_dTdz, min\_N2, additional\_mask\_dTdz}}

Minimum $dT/dz$, $N^2$ required for valid computation of $\chi, K_T, J_q^t$

The Seabird SBE-37 datasheet says (\url{http://www.seabird.com/sbe37si-microcat-ctd})
\begin{itemize}
\item T is accurate to \SI{2e-3}{\celsius}
\item conductivity is accurate to \SI{3e-3}{psu} (approx!)
\end{itemize}

i.e.,
\begin{itemize}
\item dT/dz is accurate to (2x2e-3)/dz
\item dS/dz is accurate to (2x3e-3)/dz
\end{itemize}

\subsection{Sensor deaths: \texttt{T1death, T2death; nantimes\{1,2,3\}}}

\subsection{Averaging: \texttt{avgwindow, avgvalid}}

\subsection{Orientation / sensed-volume-flushing}
\subsection{Background flow speed: \texttt{min\_inst\_spd, min\_spd, additional\_mask\_spd}}

\subsection{Maximum thresholds on $\epsilon, \chi, K_T, J_q^t$}

\subsection{Deglitching: \texttt{deglitch\_window, deglitch\_nstd
}}

\section{Useful Functions}
\subsection{ChooseEstimates}
\subsection{chiold (variable)}
combine\_turbulence saves the \emph{unmasked} chi structure as chiold after calculating Kt, Jqt but before doing any processing. Useful in checking masking.

\subsection{TestMask}
Lets you quickly iterate through various thresholds for a particular criterion.

Throws up a figure with histograms of counts to compare.

Example usage:
\begin{lstlisting}
TestMask(chi, abs(chi.dTdz), '<', [1e-4, 3e-4, 1e-3], 'Tz');\\
\end{lstlisting}

This will iterate and mask using chi.dTdz < 1e-4, then chi.dTdz < 3e-4 and finally chi.dTdz < 1e-3. Each iteration is \textbf{independent} of the previous one.

\subsection{DebugPlots}
Usually, you want to see the effect of different masking thresholds in a small subset of the time series of χ, ε etc.

Example usage:
\begin{lstlisting}
chi = chiold; \% reset to unmasked structure\\
\% apply 3 different thresholds\\
chi1 = ApplyMask(chi, abs(chi.dTdz), '<', 1e-4, 'T\(_{\text{z}}\) < 1e-4');\\
chi2 = ApplyMask(chi, abs(chi.dTdz), '<', 1e-3, 'T\(_{\text{z}}\) < 1e-3');\\
chi3 = ApplyMask(chi, abs(chi.dTdz), '<', 2e-3, 'T\(_{\text{z}}\) < 2e-3');\\
\vspace*{1em}
t0 = datetime(2016, 12, 10);\\
t1 = datenum(2016, 12, 12);\\
tavg = 600;\\
\vspace*{1em}
hf = figure;\\
\% plot 10 minute averages (tavg) of quantities in structure chi\\
\% between [t0, t1] and label them as 'raw' in figure window hf\\
DebugPlots(hf, t0, t1, chi, 'raw', tavg);\\
\vspace*{1em}
\% compare different masking; label appropriately\\
DebugPlots(hf, t0, t1, chi1, '1e-4', tavg);\\
DebugPlots(hf, t0, t1, chi2, '1e-3', tavg);\\
DebugPlots(hf, t0, t1, chi3, '2e-3', tavg);\\
\end{lstlisting}

\subsection{DebugRawData}
The idea is to connect Turb.() to the raw-ish data.

Requires structure T from temp.mat.

\subsection{Histograms2D}

%%% Local Variables:
%%% mode: latex
%%% TeX-master: "docs"
%%% End:
