\chapter{\texttt{data base}}
This chapter describes how to use the database functionality whoAmI?

At the beginning, there can be found two functions in the mfile directory of your unit directory:
\begin{itemize}
   \item \textit{whoAmI\_generate.m}
   \item \textit{whoAmI\_feedback2database.m}
\end{itemize}
Then you can use \textit{whoAmI\_generate.m} to generate a third m-file, which
is called \textit{whoAmI.m}.  \textit{whoAmI.m} is the function that contains
a complete list of all parameters that are saved in the data base that refer to
your particular unit. 

You may access these parameters by calling \textit{whoAmI.m}, which outputs
a structure containing all these parameters.  This structure can then be used
by the processing routines to get unit specific information.

If you want to change any of the parameters, you may go to \textit{whoAmI.m} and
change the corresponding parameter in the matlab code. Note that these changes are only 
local and do not immediately affect the corresponding data base entries. 
If you want to write your local changes permanently to the data base you can use
\textit{whoAmI\_feedback2database.m}. 



%%% Local Variables:
%%% mode: latex
%%% TeX-master: "docs"
%%% End:
