\chapter{Processing dissipation rates, $\epsilon$, from pitot}
This chapter explains how to process pitot tube data to obtain 
dissipation rates. 

It is assumed that the pitot tube is already calibrated such 
that a file \textit{{../calib/header\_p.mat}} exists, containing 
a structure $W$. $W$ should have the following field, 
\begin{itemize}
   \item W.Ps     -  Pressure dependence  (lab calib)
   \item W.Pd     -  dynamic pressure calibration( lab calib)
   \item W.T      -  temperature dependence (lab calib) 
   \item W.tilt   -  tilt dependence (lab calib)
   \item W.V0     -  V0 offset (determined by user)
   \item W.T0     -  mean temperature corresponding to V0 (determined by user)
   \item W.P0     -  mean pressure corresponding to V0 (determined by user)
\end{itemize}

\section{\textit{{do\_pitot\_eps}}}
\textit{{do\_pitot\_eps}} is the main driver for processing the raw data to get
dissipation rates from pitot data. It generates a file: \textit{../proc/pitot\_eps.mat},
containing a structure $Peps$ with all relevant data.

There are a couple of option that can be specified:
\begin{itemize}
   \item {save\_spec}  :  shall the spectra be saved (this increases the size of pitot\_eps a lot!!!, by adding Peps.k and Peps.D\_k )
   \item spec\_length  : is the length of the spectrum used to fit, (default is 2 sec)
   \item frange        : specify the rage to perform the fit in
\end{itemize}

If you save the spectrum (save\_spec $= 1$) than you can use peps\_spec\_plot.m 
to look at averaged or individual spectra.

\subsection{dependent files}
   here is a list of the dependent files of \textit{{do\_pitot\_eps}}
\begin{itemize}
   \item ./chipod\_gust/software/pitot/generate\_pitot\_eps.m
   \item ./chipod\_gust/software/pitot/proc\_pitot\_dissipation.m
   \item ./chipod\_gust/software/pitot/dissipation\_spec.m
   \item ./chipod\_gust/software/pitot/nasmyth\_G1.m
   \item ./chipod\_gust/software/pitot/nasmyth2Dk.m
   \item ./chipod\_gust/software/pitot/pitot\_calibrate.m
   \item ./chipod\_gust/software/pitot/flag\_Peps.m
\end{itemize}

\subsection{flag\_Peps.m}
This function generates a couple of flagging masks for the structure $Peps$ and
flags the Peps.eps with the product of these masks. 
The structure Peps does however always contain an unmasked time series of $\epsilon$,
in case you want to apply different masks. 
flag\_Peps.m is designed to mask high frequency spectra 
for moored GusTs in a surface wave contaminated environment.
If you have a different deployment or longer spectra or different environments, you
should use a different set of masks applied on the unmasked field Peps.eps\_nomaks.

